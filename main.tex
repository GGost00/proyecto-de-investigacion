\documentclass{article}
\usepackage[utf8]{inputenc}
\usepackage[spanish]{babel}
\usepackage{listings}
\usepackage{graphicx}
\graphicspath{ {images/} }
\usepackage{cite}

\begin{document}

\begin{titlepage}
    \begin{center}
        \vspace*{1cm}
            
        \Huge
        \textbf{Nociones de la memoria del computador}
            
        \vspace{0.5cm}
        \LARGE
        Subtítulo
            
        \vspace{1.5cm}
            
        \textbf{Gabriel Andres Restrepo Izquierdo}
            
        \vfill
            
        \vspace{0.8cm}
            
        \Large
        Despartamento de Ingeniería Electrónica y Telecomunicaciones\\
        Universidad de Antioquia\\
        Medellín\\
        Septiembre de 2020
            
    \end{center}
\end{titlepage}

\tableofcontents

\section{definicion de que es la memoria de un computador.}

\section{tipos de memorias que conozco} \label{contenido}

\section{Describción la manera como se gestiona la memoria en un computador.}

\section{¿qué hace que una memoria sea más rápida que otra? ¿Por qué esto es importante?}
Esta sección es para ver qué pasa con los comandos 
que definen texto

El paquete también agrega un comportamiento especial 
a <<estas marcas para hacer citas textuales>> tal como 
lo indican las reglas de la RAE. \cite{dirac}

citamos un nuevo autos \cite{knuth-fa}
\begin{lstlisting}
#include <stdio.h>
#define N 10
/* Block
 * comment */

int main()
{
    int i;

    // Line comment.
    puts("Hello world!");
    
    for (i = 0; i < N; i++)
    {
        puts("LaTeX is also great for programmers!");
    }

    return 0;
}
\end{lstlisting}

A continuación se presenta el logo de C++ Figura (\ref{fig:cpplogo})

\begin{figure}[h]
\includegraphics[width=4cm]{cpplogo.png}
\centering
\caption{Logo de C++}
\label{fig:cpplogo}
\end{figure}

En la sección de teoremas (\ref{contenido})

\section{Conclusión} \label{conclulsion}

\bibliographystyle{IEEEtran}
\bibliography{references}

\end{document}
