\documentclass{article}
\usepackage[utf8]{inputenc}
\usepackage[spanish]{babel}
\usepackage{listings}
\usepackage{graphicx}
\graphicspath{ {images/} }
\usepackage{cite}

\begin{document}

\begin{titlepage}
    \begin{center}
        \vspace*{1cm}
            
        \Huge
        \textbf{Nociones de la memoria del computador}
            
        \vspace{0.5cm}
        \LARGE
        Subtítulo
            
        \vspace{1.5cm}
            
        \textbf{Gabriel Andres Restrepo Izquierdo}
            
        \vfill
            
        \vspace{0.8cm}
            
        \Large
        Despartamento de Ingeniería Electrónica y Telecomunicaciones\\
        Universidad de Antioquia\\
        Medellín\\
        Septiembre de 2020
            
    \end{center}
\end{titlepage}

\tableofcontents

\newpage

\section{Definicion de que es la memoria de un computador.}
Según lo que leí y entiendo del documento la memoria de un computador es utilizada para trabajar la información de manera segura y más rápida en la cual el usuario puede trabajar sus documentos de manera fácil y hacerle modificaciones sin alterar el documento original.

Hay varios tipos de memorias unos más rápidos que otros o con más o menos espacio que otros los cuales almacenan información temporal y más recurrente para poder acceder a ella de una manera más sencilla y no tener que buscarla de nuevo en el disco duro.

En la memoria se guardan datos los cuales se pueden alterar de manera independiente, pero si se cierra un programa alterado sin guardar nuevamente en el disco duro la información desaparece ya que la memoria solo guarda archivos de manera temporal para el mejorar su rendimiento.  


\section{Tipos de memorias que conozco} \label{contenido}
Antes de leer el documento conocía dos tipos de memoria la memoria RAM y el disco duro.

 \textbf{Memoria RAM}: La memoria RAM es uno de los tipos de memoria más importantes en un computador. Ella guarda información en celdas en forma de unos y ceros, la información guardada en dicha memoria se puede acceder de forma rápida y sencilla ya que no guarda la información de forma serial si no aleatoria.\cite{sistema}
 
 \textbf{Disco duro}: Un disco duro es aquel que sirve para guardar información de forma casi permanente, este tipo de memoria es necesario para que la información no se pierda y se mantenga de manera segura. Todos los computadores tienen una integrada en su placa madre ya que sin ella la información quedaría desechable cada vez que el computador se apagara. \cite{Disco_Duro}

\section{Describción la manera como se gestiona la memoria en un computador.}
La gestión de la memoria se hace a través de un microcontrolador que interviene en cada operación y mide la velocidad con la que se transporta la información la cual mide en MHz. El controlados se puede encontrar en una de estas dos partes:

•	 Un chip situado en la tarjeta madre entre los módulos de memoria y la CPU

•	Dentro del microprocesador 

Este microcontrolador se puede denominar como MCH (Memory Controller Hub o en español, Centro de Control de Memoria).


\section{¿Qué hace que una memoria sea más rápida que otra?}
Lo que hace que una memoria sea más rápida que otra es por la frecuencia y la latencia que tenga si una memoria tiene más frecuencia que latencia es mucho mejor porque el valor resultante de la velocidad es mucho más beneficioso que si tiene mayor latencia y poca frecuencia porque la frecuencia es la  cantidad de veces por segundo que se tarda en hacer una operación y la latencia es el tiempo que se demora la memoria mientras realiza una operación y la velocidad se calcula de la siguiente forma: \cite{frecuencia_latencia}

$$\frac{latencia(CAS)}{frecuencia(MT/s )}\cdot2\cdot1024$$ \cite{formula}
Al hacer ese cálculo el que de menor valor es el que mejor rendimiento ofrece, aunque no siempre es cierto.\cite{frecuencia_latencia}
\subsection{¿Por qué esto es importante?}

La velocidad determina la rapidez a la que puede trabajar una memoria y afecta junto a su bus de datos su ancho de banda una mayor velocidad es necesaria dado que para realizar tareas importantes uno necesita un rendimiento mejor y que se haga con una buena velocidad ya que las operaciones de almacenar, borrar y re almacenar nueva información y datos se hará mucho más rápidamente lo que puede marcar la diferencia en su rendimiento 
\cite{memoria_imp}





\newpage
\bibliographystyle{IEEEtran}
\bibliography{references}

\end{document}
